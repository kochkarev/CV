% vim: set textwidth=120:

% Example CV based on the 1.5-column-cv template. Main features:
% * uses the Roboto font family and IcoMoon icon set;
% * doesn't use colours, different font weights are used instead for styling;
% * because the CV fits on one page, header and footer is empty, since there isn't much useful info to put there;
% * includes a photo.
\documentclass[a4paper,10pt]{article}


% package imports
% ---------------

\usepackage[british, russian]{babel} % for correct language and hyphenation and stuff
\usepackage{calc}           % for easier length calculations (infix notation)
\usepackage{enumitem}       % for configuring list environments
\usepackage{fancyhdr}       % for setting header and footer
\usepackage{fontspec}       % for fonts
\usepackage{geometry}       % for setting margins (\newgeometry)
\usepackage{graphicx}       % for pictures
\usepackage{microtype}      % for microtypography stuff
\usepackage{xcolor}         % for colours


% margin and column widths
% ------------------------

% margins
\newgeometry{left=15mm,right=15mm,top=15mm,bottom=15mm}

% width of the gap between left and right column
\newlength{\cvcolumngapwidth}
\setlength{\cvcolumngapwidth}{3.5mm}

% left column width
\newlength{\cvleftcolumnwidth}
\setlength{\cvleftcolumnwidth}{44.5mm}

% right column width
\newlength{\cvrightcolumnwidth}
\setlength{\cvrightcolumnwidth}{\textwidth-\cvleftcolumnwidth-\cvcolumngapwidth}

% set paragraph indentation to 0, because it screws up the whole layout otherwise
\setlength{\parindent}{0mm}


% style definitions
% -----------------
% style categories explanation:
% * \cvnameXXX is used for the name;
% * \cvsectionXXX is used for section names (left column, accompanied by a horizontal rule);
% * \cvtitleXXX is used for job/education titles (right column);
% * \cvdurationXXX is used for job/education durations (left column);
% * \cvheadingXXX is used for headings (left column);
% * \cvmainXXX (and \setmainfont) is used for main text;
% * \cvruleXXX is used for the horizontal rules denoting sections.

% font families
\defaultfontfeatures{Ligatures=TeX} % reportedly a good idea, see https://tex.stackexchange.com/a/37251

\newfontfamily{\cvnamefont}{Roboto Medium}
\newfontfamily{\cvsectionfont}{Roboto Medium}
\newfontfamily{\cvtitlefont}{Roboto Regular}
\newfontfamily{\cvdurationfont}{Roboto Light Italic}
\newfontfamily{\cvheadingfont}{Roboto Regular}
\setmainfont{Roboto Light}

% colours
\definecolor{cvnamecolor}{HTML}{000000}
\definecolor{cvsectioncolor}{HTML}{000000}
\definecolor{cvtitlecolor}{HTML}{000000}
\definecolor{cvdurationcolor}{HTML}{000000}
\definecolor{cvheadingcolor}{HTML}{000000}
\definecolor{cvmaincolor}{HTML}{000000}
\definecolor{cvrulecolor}{HTML}{000000}

\color{cvmaincolor}

% styles
\newcommand{\cvnamestyle}[1]{{\Large\cvnamefont\textcolor{cvnamecolor}{#1}}}
\newcommand{\cvsectionstyle}[1]{{\normalsize\cvsectionfont\textcolor{cvsectioncolor}{#1}}}
\newcommand{\cvtitlestyle}[1]{{\large\cvtitlefont\textcolor{cvtitlecolor}{#1}}}
\newcommand{\cvdurationstyle}[1]{{\small\cvdurationfont\textcolor{cvdurationcolor}{#1}}}
\newcommand{\cvheadingstyle}[1]{{\normalsize\cvheadingfont\textcolor{cvheadingcolor}{#1}}}


% inter-item spacing
% ------------------

% vertical space after personal info and standard CV items
\newlength{\cvafteritemskipamount}
\setlength{\cvafteritemskipamount}{5mm plus 1.25mm minus 1.25mm}

% vertical space after sections
\newlength{\cvaftersectionskipamount}
\setlength{\cvaftersectionskipamount}{2mm plus 0.5mm minus 0.5mm}

% extra vertical space to be used when a section starts with an item with a heading (e.g. in the skills section),
% so that the heading does not follow the section name too closely
\newlength{\cvbetweensectionandheadingextraskipamount}
\setlength{\cvbetweensectionandheadingextraskipamount}{1mm plus 0.25mm minus 0.25mm}


% intra-item spacing
% ------------------

% vertical space after name
\newlength{\cvafternameskipamount}
\setlength{\cvafternameskipamount}{3mm plus 0.75mm minus 0.75mm}

% vertical space after personal info lines
\newlength{\cvafterpersonalinfolineskipamount}
\setlength{\cvafterpersonalinfolineskipamount}{2mm plus 0.5mm minus 0.5mm}

% vertical space after titles
\newlength{\cvaftertitleskipamount}
\setlength{\cvaftertitleskipamount}{1mm plus 0.25mm minus 0.25mm}

% value to be used as parskip in right column of CV items and itemsep in lists (same for both, for consistency)
\newlength{\cvparskip}
\setlength{\cvparskip}{0.5mm plus 0.125mm minus 0.125mm}

% set global list configuration (use parskip as itemsep, and no separation otherwise)
\setlist{parsep=0mm,topsep=0mm,partopsep=0mm,itemsep=\cvparskip}


% CV commands
% -----------

% creates a "personal info" CV item with the given left and right column contents, with appropriate vertical space after
% @param #1 left column content (should be the CV photo)
% @param #2 right column content (should be the name and personal info)
\newcommand{\cvpersonalinfo}[2]{
    % left and right column
    \begin{minipage}[t]{\cvleftcolumnwidth}
        \vspace{0mm} % XXX hack to align to top, see https://tex.stackexchange.com/a/11632
        \raggedleft #1
    \end{minipage}% XXX necessary comment to avoid unwanted space
    \hspace{\cvcolumngapwidth}% XXX necessary comment to avoid unwanted space
    \begin{minipage}[t]{\cvrightcolumnwidth}
        \vspace{0mm} % XXX hack to align to top, see https://tex.stackexchange.com/a/11632
        #2
    \end{minipage}

    % space after
    \vspace{\cvafteritemskipamount}
}

% typesets a name, with appropriate vertical space after
% @param #1 name text
\newcommand{\cvname}[1]{
    % name
    \cvnamestyle{#1}

    % space after
    \vspace{\cvafternameskipamount}
}

% typesets a line of personal info beginning with an icon, with appropriate vertical space after
% @param #1 parameters for the \includegraphics command used to include the icon
% @param #2 icon filename
% @param #3 line text
\newcommand{\cvpersonalinfolinewithicon}[3]{
    % icon, vertically aligned with text (see https://tex.stackexchange.com/a/129463)
    \raisebox{.5\fontcharht\font`E-.5\height}{\includegraphics[#1]{#2}}
    % text
    #3

    % space after
    \vspace{\cvafterpersonalinfolineskipamount}
}

% creates a "section" CV item with the given left column content, a horizontal rule in the right column, and with
% appropriate vertical space after
% @param #1 left column content (should be the section name)
\newcommand{\cvsection}[1]{
    % left and right column
    \begin{minipage}[t]{\cvleftcolumnwidth}
        \raggedleft\cvsectionstyle{#1}
    \end{minipage}% XXX necessary comment to avoid unwanted space
    \hspace{\cvcolumngapwidth}% XXX necessary comment to avoid unwanted space
    \begin{minipage}[t]{\cvrightcolumnwidth}
        \textcolor{cvrulecolor}{\rule{\cvrightcolumnwidth}{0.3mm}}
    \end{minipage}

    % space after
    \vspace{\cvaftersectionskipamount}
}

% creates a standard, multi-purpose CV item with the given left and right column contents, parskip set to cvparskip
% in the right column, and with appropriate vertical space after
% @param #1 left column content
% @param #2 right column content
\newcommand{\cvitem}[2]{
    % left and right column
    \begin{minipage}[t]{\cvleftcolumnwidth}
        \raggedleft #1
    \end{minipage}% XXX necessary comment to avoid unwanted space
    \hspace{\cvcolumngapwidth}% XXX necessary comment to avoid unwanted space
    \begin{minipage}[t]{\cvrightcolumnwidth}
        \setlength{\parskip}{\cvparskip} #2
    \end{minipage}

    % space after
    \vspace{\cvafteritemskipamount}
}

% typesets a title, with appropriate vertical space after
% @param #1 title text
\newcommand{\cvtitle}[1]{
    % title
    \cvtitlestyle{#1}

    % space after
    \vspace{\cvaftertitleskipamount}
    % XXX need to subtract cvparskip here, because it is automatically inserted after the title "paragraph"
    \vspace{-\cvparskip}
}


% header and footer
% -----------------

% set empty header and footer
\pagestyle{empty}



% preamble end/document start
% ===========================

\begin{document}


% personal info
% -------------

\cvpersonalinfo{
    % photo
    \includegraphics[height=36mm]{me.jpg}
}{
    % name
    \cvname{Кочкарев Алексей Вячеславович}

    % address
    \cvpersonalinfolinewithicon{height=4mm}{072-location.pdf}{
        Москва, Россия
    }

    % phone number
    \cvpersonalinfolinewithicon{height=4mm}{067-phone.pdf}{
        +7 985 459 1427
    }

    % email address
    \cvpersonalinfolinewithicon{height=4mm}{070-envelop.pdf}{
        alexey.kochkarev@gmail.com
    }

    % LinkedIn account
    \cvpersonalinfolinewithicon{height=4mm}{458-linkedin.pdf}{
        linkedin.com/in/kochkarev
    }
    
    \cvpersonalinfolinewithicon{height=4mm}{github_logo.pdf}{
        github.com/kochkarev
    }

    % date of birth
    Дата рождения: 19 июня 1996
}


% work experience
% ---------------

\cvsection{ОПЫТ РАБОТЫ}

% Fake Company 2
\cvitem{
    \cvdurationstyle{Июль 2019 -- н.в.}
}{
    \cvtitle{Разработчик C++}

    ПАО Сбербанк

    \begin{itemize}[leftmargin=*]
        \item Эмулирование CEN/XFS RATM Nautilus.\\ Технологии: C++17 (STL, boost, boost::asio).
    \end{itemize}
}


% education
% ---------

\cvsection{ОБРАЗОВАНИЕ}

% master's
\cvitem{
    \cvdurationstyle{2016 -- 2020}
}{
    \cvtitle{Бакалавр}

    Московский Государственный Университет им. М. В. Ломоносова \\ Факультет Вычислительной Математики и Кибернетики

    \begin{itemize}[leftmargin=*]
        \item Тема дипломной работы: Метод автоматического анализа минерального\\ состава руд по фотографиям аншлифов
    \end{itemize}
}

% education
% ---------

\cvsection{Сертификаты}

% master's
\cvitem{
    \cvdurationstyle{Октябрь 2019}
}{
    \cvtitle{Нейронные сети и компьютерное зрение\\ (Samsung Research Russia)}

}
\cvitem{
    \cvdurationstyle{Апрель 2020}
}{
    \cvtitle{Программирование на языке C++ (Computer Science Center)}

    \begin{itemize}[leftmargin=*]
        \item Сертификат с отличием
    \end{itemize}
}
\cvitem{
    \cvdurationstyle{Май 2020}
}{
    \cvtitle{Программирование на языке C++ (продолжение)\\ (Computer Science Center)}

    \begin{itemize}[leftmargin=*]
        \item Сертификат с отличием
    \end{itemize}
}



% skills
% ------

\cvsection{Технические навыки}

\vspace{\cvbetweensectionandheadingextraskipamount}

\cvitem{
    \cvheadingstyle{Языки программирования}
}{
    
    \begin{itemize}
    \item C/C++ (boost)
    \item Python 3.7 (numpy, opencv, pillow, tensorflow, keras, pytorch)
    \item C\# (.NET Framework, .NET Core, ASP.NET)
    \end{itemize}

    
}

\cvitem{
    \cvheadingstyle{Математические дисциплины}
}{
    \begin{itemize}
        \item Линейная алгебра и аналитическая геометрия
        \item Дифференциальные уравнения (в т. числе в частных производных)
        \item Численные методы
    \end{itemize}
}

\cvitem{
    \cvheadingstyle{Компьютерное зрение}
}{
    \begin{itemize}
    \item Базовые знания в обработке изображений: \\
    преобразование Фурье, метод Канни, линейные и нелинейные фильтры, свертка
    \item Классификация и мультиклассовая сегментация с использованием \\ сверточных нейронных сетей
    \item Трекинг объектов
    \end{itemize}
}

\cvitem{
    \cvheadingstyle{Технологии и прочее}
}{
    
    \begin{itemize}
        \item Git
        \item Jenkins
        \item UNIX
        \item \LaTeX
    \end{itemize}
}

\newpage
\cvsection{Проекты}

\vspace{\cvbetweensectionandheadingextraskipamount}

\cvitem{
    \cvdurationstyle{Сентябрь 2019 -- н.в}
}{
    \cvtitle{Метод автоматического анализа минерального сотава руд \\ по фотографиям аншлифов}

    \begin{itemize}[leftmargin=*]
        \item ПО для многоклассовой сегментации минералов на реальных фотографиях аншлифов. Комбинация сверточной нейронной сети, основанной на UNet и классических методов обработки изображений. Планируется доклад на ГрафиКон.
        
    \end{itemize}
}

\cvitem{
    \cvdurationstyle{Апрель -- Май 2018}
}{
    \cvtitle{Парсер модельного языка}
    \begin{itemize}[leftmargin=*]
        \item Лексический, ситнаксический и семантический анализаторы, метод рекурсивного спуска.
        
    \end{itemize}
}


\cvsection{Личные навыки}

\vspace{\cvbetweensectionandheadingextraskipamount}

\cvitem{
    \cvheadingstyle{}
}{
    
    \begin{itemize}
        \item Прекрасно работаю в команде
        \item Умею обрабатывать и усваивать большие объемы информации в сжатые сроки
        \item Люблю разбираться в новых технологиях
        \item Умею применять на практике теоретические знания
    \end{itemize}

    
}


% languages
\cvitem{
    \cvheadingstyle{Знание языков}
}{
    
    \begin{itemize}
        \item Русский -- родной
        \item English -- Advanced (C1)
        \item Deutsch -- Grundstufe
    \end{itemize}


}

% additional info
% ---------------

\cvsection{Достижения}

\vspace{\cvbetweensectionandheadingextraskipamount}

\cvitem{
    \cvheadingstyle{}
}{
    \begin{itemize}
        \item Был разработан алгоритм балансировки данных, позволяющий решать \\ проблему несбалансированности данных в задачах глубокого обучения.
    \end{itemize}
}


\end{document}